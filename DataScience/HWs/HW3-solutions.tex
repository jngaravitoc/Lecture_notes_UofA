\documentclass[14pt]{article}
\usepackage{amsmath}
\usepackage{listings} % For writing code see http://ctan.org/pkg/listings
\usepackage{graphicx}
\usepackage{float}
\usepackage[margin=1.0in]{geometry}
\usepackage{mdframed}
\usepackage{hyperref}

\title{CS665 HW3: Solutions}
\author{Nicolas Garavito-Camargo}
\begin{document}
\maketitle

\section{Part I}

\begin{mdframed}
Let $x_1, $ be $n$ points in d-dimensional  space  and  let
$X$ be the $n\times d$ matrix whose rows are the
$n$ points.  Suppose we know only the matrix $D$
of pairwise distances between points and not the coordinates of the points
themselves.  The set of points $x x_n$  giving  rise  to  the  distance  matrix
$D$ is  not  unique  since  any  translation,
rotation,  or  reflection  of  the  coordinate  system  leaves  the
distances  invariant.   Fix  the origin  of  the  coordinate  system  so  that  the  centroid  of  the
set  of  points  is  at  the  origin. That is, $\sum_i=1^n x_i = 0$
\begin{enumarate}
\item Show that the elements of $XX^T$ are given by:

\begin{equation}
x_i x_j^T = - \dfrac{1}{2} \left[ d_{ij}^2 - \dfrac{1}{n} \sum_{j=1}^n
d_{ij}^2 - \dfrac{1}{n}\sum_{i=1}^n d_{ij}^2 + \dfrac{1}{n^2}
\sum_{i=1}^2 \sum_{j=1}^n d_{ij}^2 \right]
\end{equation}

\item Describe an algorithm for determining the matrix $X$ whose rows
are the $x_i$.

\end{enumarete}

\end{mdframed}

\textbf{Solution:}\\

\textbf{1.} The distance matrix $D$ with components $d_{ij}^2$ is defined as:

\begin{equation}
d_{ij}^{2} = (\sum_{i} x_{i})^2 + (\sum_{j} x_{j})^2 - 2 \sum_{i,j} x_i
x_j
\end{equation}

Which at the origin of the coordinate system it would be:

\begin{equation}\label{eq:dij}
d_{ij}^{2} =  - 2 \sum_{i,j} x_i x_j
\end{equation}

On the other hand the matrix $xx^T$ is:

\begin{equation}
x x^T_{ij} = (x_i - \dfrac{1}{n}\sum_i x_i )(x_j - \dfrac{1}{n}\sum_j
x_j )
\end{equation}

Doing the products the components of the matrix are:

\begin{equation}\label{eq:xxt}
xx^T_{ij} =  \sum_{ij}^n x_i x_j - \dfrac{1}{n} \sum_i x_i x_j -
\dfrac{1}{n}\sum_j x_i x_j + \dfrac{1}{n^2} \sum_{ij} x_i x_j 
\end{equation}

Using the results of Eq. \ref{eq:dij} into Eq. \ref{eq:xxt} $xx^T$ is:

\begin{equation}
xx^T_{ij} =  -\dfrac{1}{2} \left[  d_{ij}^2 - \dfrac{1}{n}
\sum_i d_{ij}^2 -
\dfrac{1}{n}\sum_j d_{ij}^2 + \dfrac{1}{n^2} \sum_{ij} d_{ij}^2 \right]
\end{equation}

Which is the desired result.\\

\textbf{2.} 

In order to determine $X$ from $D$ we need to center the matrix $D$
an

\section{Part II}

\end{document}

