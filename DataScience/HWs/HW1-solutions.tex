\documentclass[14pt]{article}
\usepackage{amsmath}
\usepackage{listings} % For writing code see http://ctan.org/pkg/listings
\usepackage{graphicx}
\usepackage{float}
\usepackage[margin=1.0in]{geometry}
\usepackage{mdframed}

\title{CS665 HW1: Solutions}
\author{Nicolas Garavito-Camargo}
\begin{document}
\maketitle

\begin{mdframed}

1. Given a stream of symbols $a_1,a_2,…,a_n$ each an integer in
$1,…,m,$ give an algorithm that will select one symbol uniformly
 at random from the stream. How much memory does your algorithm require?

\end{mdframed}

\textbf{Solution:}\\

A random probability to select a given symbol of the stream
would be $P = \dfrac{1}{n}$ where $n$ in the number of symbols in the
stream. If we are reading one symbol at a time, the probability of
reading the symbol $j$ would
be $P = \dfrac{1}{\sum j}$ where $j$ is the index of the symbol. Therefore 
the probability of selecting the symbol $j+1$ is $P_{j+1} =
\dfrac{1}{\sum (j+1)}$.\\


\begin{mdframed}
2. Give an algorithm to select an $a_i$ from a stream of symbols
$a_1,a_2,…,a_n$ with probability proportional to $a^2_i$.\\
\end{mdframed}

\textbf{Solution:}\\


In this case the probability of selecting the symbol $a_j$ of the stream 
would be $P = \dfrac{1}{S_{j}}$ where $S_j = \sum a_j^2$.

\begin{mdframed}
3. How would one pick a random word from a very large book where the
probability of picking a word is proportional to the number of
occurrences of the word in the book? \\
\end{mdframed}

\textbf{Solution:}\\

To tackle this problem I am going to assume that the stream is very
large with length $n$ such that a simple counting algorithm can't
work. Therefore I propose the following algorithm:

\begin{itemize}
\item Initialize a counter for each new word that appears in your stream.

\item As you read the words make a hash table that stores the number of
words that you have accepted
\item Assign a probability to each word to be picked based on the number of
appearances in the hash table. The probability is going to be $P_w =
n_w/m$, where $M$ is the total words read (this comes from the hash
table). $n_w$ is the number of apperences of the word, (this also
comes from the hash table)
\item Because the distribution of the number of appereances of the
words should be a Gaussian distribution the probability $P_w$ is going to
converge as we read words. As such one can stop counting words as soon
as the change of $P_w$ is small as you read words, say $\Delta_P_w=1\%$.


\end{itemize}

Keep track of the appearances of each element and the probability is
going to be $P = \dfrac{1}{\sum  }$

\begin{framed}
4. Consider a matrix where each element has a probability of being
selected. Can you select a row according to the sum of probabilities
of elements in that row by just selecting an element according to its
probability and selecting the row that the element is in?
\end{framed}

\textbf{Solution:}\\





\end{document}

