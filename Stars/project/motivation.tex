\section*{Motivation:}

The aim of this project is to study oscillations in thin disks, in
particular the dynamics of the oscillations and the what are the
causes of the excitations. A deep understanding of the physics behind
these oscillations would allow to study effect as the quasi-periodic
oscillations. Also variations in the X-ray flux from black holes and
AGNs might be explained. To this aim we follow mainly the review
article by \textbf{Kato 2001}.

In accretion disks the main forces are the gravitational force and the
centrifugal force. When these two forces are in equilibrium the disk
is stable. The gravitational force is the one from the central object
of the accretion disk.

The excitation mechanisms of disk oscillations are: XXX, XXX, XXX, and viscous
processes. Viscosity is the major source of heating in the disk, also
the azimuthal force caused by viscosity produces angular momentum
transport in the radial direction. The first one can be seen as a
thermal process while the second one is a dynamical process.

Accretion disks are collisional and non-selfgravitating.
