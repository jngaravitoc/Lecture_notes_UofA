\section*{The dispersion relation of disks:}

\begin{equation}
\tilde{\omega} = \omega - m\Omega
\end{equation}

Epicyclic frequency:

\begin{equation}
\kappa^2 = 2\Omega \left( 2\Omega + r \dfrac{d\Omega}{dr}  \right)
\end{equation}

Dispersion relation:

\begin{equation}\label{eq:dispersion}
(\tilde{\omega}^2 - \kappa^2)(\tilde{\omega}^2 - n\Omega_k^2) =
\tilde{\omega}^2 c_s^2 k_r^2
\end{equation}

To understand this dispersion relation it is better to study the limit
cases:

In the case of oscillations in the plane if the disk
$n=0$ Eq.\ref{eq:dispersion} reduced to:

\begin{equation}\label{eq:disp1}
\tilde{\omega}^2 = \kappa^2 + k_r^2 c_s^2
\end{equation}

This condition is known as \textbf{inertial-acoustic} waves
and corresponds to the oscillations of a fluid element that was
displaced in the radial direction, the oscillations arise to the
resorting forces that brings back the fluid to the initial position.
The oscillation frequency is the epicyclic frequency $\kappa(r)$ first
term in the right part of equation \ref{eq:disp1}. While the second
term corresponds to acoustic oscillations due to the restoring force
from compressible fluids.

Now in the long-wavelength limit ($k_r=0$) Eq.\ref{eq:disp1} is
reduced to:

\begin{equation}
\tilde{\omega}^2 = \kappa^2
\end{equation}

\begin{equation}
\tilde{\omega}^2 = n \Omega_K^2
\end{equation}

Which corresponds to vertical oscillations in the disk, due to a
perturbation of a fluid element in the vertical direction. The
vertical component of the gravitational force in the restore force
that returns the fluid element to the plane of the disk. The frequency
of this oscillations is $\Omega_K$.

This two oscillations are coupled in the form $(\tilde{\omega}^2 -
\kappa^2)(\tilde{\omega}^2 - n\Omega_k^2)$ in the dispersion relation
Eq.\ref{eq:dispersion}. Vertical oscillations induce perturbations
in the radial direction due to the inhomogeneities in the disk. The
coupling is stronger when the radia wavelength is shorter and the
acoustic speed is faster.

The solutions for Eq.\ref{eq:dispersion} are:

\begin{equation}\label{eq:sol}
\tilde{\omega}^2 = \dfrac{(n\Omega_k^2 + \kappa^2 + c_s^2 K_r^2) \pm
\sqrt( - 4\kappa^2n\Omega_k^2)}{2}
\end{equation}

The modes with the $+$ sign in Eq.\ref{eq:sol} are called
\textbf{p-modes} while the solutions with $-$ are called
\textbf{g-modes}

\subsection{Relativistic effects on the Dispersion Relation}

When general relativistic effects are taking into account 

\begin{equation}
\kappa^2 = \dfrac{GM}{r^3}\left( 1 + \dfrac{a}{\hat{r}^{3/2}}
\right)^{-2} \left(1 - \dfrac{6}{\hat{r}} + \dfrac{8a}{\hat{r}^{3/2}}
- \dfrac{3a^2}{\hat{r}^2}  \right)
\end{equation}

Where $a$ is a dimensionless parameter specifying the amount of
angular momentum of the central object if $a=0$ the object is rotating
and if $a=1$ the central object have the maximum rotation (the case of
extreme Kerr).

\begin{equation}
\hat{r} = \dfrac{r}{GM/c^2}
\end{equation}

\begin{equation}
\Omega_{}^2 = \Omega_{K}^2 \left(1 - \dfrac{4a}{\hat{r}^{3/2} +
\dfrac{3a^2}{\hat{r}^2}} \right)
\end{equation}

\begin{equation}
\Omega_K^2 = \dfrac{GM}{r^3} \left[ 1 + \dfrac{a}{(8
\hat{r}^3)^{1/2}}\right]^{-1}
\end{equation}

\begin{equation}
(\tilde{\omega}^2 + \kappa^2)(\tilde{\omega}^2 - n \Omega_{}^2) =
\tilde{\omega}^2 c_s^2 k_{r}^2
\end{equation}
