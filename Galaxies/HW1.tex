\documentclass[14pt]{article}
\usepackage{amsmath}
\usepackage{listings} % For writing code see http://ctan.org/pkg/listings
\usepackage{graphicx}
\usepackage{float}
\usepackage[margin=0.5in]{geometry}

\title{HW \#1 ASTR 540}
\author{Nico Garavito}
\begin{document}
\maketitle

1. The declination of a star is $42^{\circ}57'$ N and its proper
motion components are $\mu_{\alpha}=-0.0374''$, $\mu_{\delta}=1.21"$.
Calculate its total proper motion. If the spectrum reveals a blueshift
of $7.6 kms^{-1}$ and the parallax is $0.376"$, calculate its space
velocity relative tp the Sun and its total proper motion at the time
of closest approach. \\

\textbf{Solution:}

\begin{table}[h]
\centering
\begin{tabular}{c}
\hline
$\delta = 42^{\circ} 57' = 42.95^{\circ}$ \\
$\mu_{\alpha} = -0.0374''$ \\
$\mu_{\delta} = 1.21''$\\
\hline
\end{tabular}
\caption{\label{tab:1}}
\end{table}

From the proper motion components resumed in table \ref{tab:1}
we can compute the total proper motion as follows:

\begin{equation}
\mu_{\delta} = \mu cos(\theta)
\end{equation}

\begin{equation}
\mu = \dfrac{\mu_{\delta}}{cos(\theta)}
\end{equation}

\begin{equation}
\mu_{\alpha} = \dfrac{\mu sin(\theta)}{cos(\delta)}
\end{equation}

\begin{equation}
tan(\theta)=\dfrac{cos(\delta) \mu_{\alpha}}{\mu_{\delta}}
\end{equation}

\begin{equation}
\theta = 1.3035^{\circ}
\end{equation}

And the total proper motion would be:

\begin{equation}
\mu = 1.21 ''/yr
\end{equation}

With the parallax we can compute the distance as follows:

\begin{equation}
d = \dfrac{1AU}{tan(0.376'')} = 180 AU /(0.376'' \pi) = 548576.612 AU
\end{equation}

The space velocity relative to the Sun would be:

\begin{equation}
V_t = \mu \cdot d = \dfrac{1.21}{3600} \dfrac{\pi}{180} 548576.12 AU
/yr = 15.255 km/s
\end{equation}

\begin{equation}
V = (V_t^2 + V_r^2)^{(1/2)} = (-0.46^2 + 7.6^2)^{1/2} = 17.03 km/s
\end{equation}

Finally the total proper motion at the time of closest approach would
be:

\begin{equation}
\mu = \dfrac{V_t}{d}
\end{equation}

Where the distance of closest approach $d_0$ would be:

\begin{equation}
d_0 = d \cdot sin(\theta) = d \cdot sin(arctan(V_t/V_r)) =  491015.38
 AU
\end{equation}

And the total proper motion at this time $t_0$ is going to be:

\begin{equation}
\mu = 17.03 kms^{-1} / 491015.38 AU = 1.5''/yr
\end{equation}

2. Smoot et al (1992) found a dipole anisotropy in COBE measurements
of the microwave background such that the background is higher by
$3.36mK$ (c.f. 2.73K average) in the direction $\alpha=11h09m$
,$\delta=-7^{\circ}$. By subtracting Galatic rotational motion of
the solar neighborhood ($v=220km/s$), determine the direction (in
celestial coordinates) and peculiar speed of the Milky Way with
respect to the microwave background. The IAU approved celestial
coodiantes of the North Galactic Pole is $192^{\circ}.85948,
27^{\circ}.12825$ and the Galactic longitude of th north celestial pole
is $123.932$.\\

\textbf{Solution:}

Using the difference in temperature of the microwave background and
the dipole anisotropy the relative velocity of the Solar neighbor with 
respect to the CMB can be calculated as:

\begin{equation}
\dfrac{T'}{T} = \left( \dfrac{c-v}{c+v}\right)^{1/2}
\end{equation}

Where $T=2.73K$ and $T'=T+3.36mK$. After some algebra I found:

\begin{equation}
v = \dfrac{c\left (1 - \left (\dfrac{T'}{T}\right )^2 \right )}
{1 + \left (\dfrac{T'}{T}\right )^2} = -369.23 km/s
\end{equation}

The direction of the CMB anisotropy in Galactic coordinates can be
computed using the coordinates transformation relation, I get
$l=263.22^{\circ}$ and $b=47.81^{\circ}$ (using \textsc{Astropy}).

Finally the component of the solar neighbor rotational velocity
$v_{\odot}$ in the
direction of the CMB dipole anisotropy $v_{\odot, CMB}$ is:

\begin{equation}
v_{\odot, CMB} = v_{\odot} cos(\theta)cos(\phi) = -146.71 km/s
\end{equation}

Finally the peculiar velocity of the Milky Way with respect to the
CMB is:

\begin{equation}
v_{MW, CMB} = -146.71+(-369.23) = -515.94 km/s
\end{equation}

3. Bahcall \& Soneira's (1980) assumed that the luminosity density of
the Galactic disk as a function of radius can be fit by an exponential
function. Using near-infrared (K-band) spacelab 2 data, they
determined a central disk luminosity of $1208  L_{odot} pc^{-2}$ and a
disk scale length $R_d=2.7 kpc$. If the distance from the Sun to the
Galactic center is $R_0=8kpc$, estimate the projected surface
brightness $\mu_K$ (in mags arcsec$^{-2}$) of the Galactic disk in
the solar neighborhood as viewed by an external observer. The absolute
magnitude of the Sun is $M_K=+3.28$.\\

\textbf{Solution:}

The absolute magnitude of the Galaxy would be:

\begin{equation}
M_{bol, gal} = M_B - 2.5Log_{10}(L_{gal}/L_{\odot}) = 5.48 -
2.5Log_{10}(1.4\times 10^{10}) = -19.58
\end{equation}

The luminosity density at the Solar neighborhood would be:

\begin{equation}
1208 L_{\odot} pc^{-2} exp(-8.0/3.0) = 83.93 L_{\odot} pc^{-2}
\end{equation}

And the projected surface brightness for an external observer 
at an angle $\theta$ would be: 

\begin{equation}
\mu_{k} = -2.5log_{10}(I^2d^2cos(\theta)) + M_{k{\odot}} + 5 log_{10}(d) -
5 = -2.5log_{10}(I^2) - 2.5log_{10}cos(\theta) + M_{k{\odot}}
\end{equation}

With the numbers given by the problem the surface brightness would be
in terms of $\theta$:

\begin{equation}
\mu_{k} = -1.59 -2.5log_{10}(cos(\theta))
\end{equation}

4. Assume for the time being that the Galaxy has no dist, and that we
are observing along a line of sight at $b=0^{\circ}$ and
$l=180^{\circ}$. We are interested in observing the most distant
solar-type stars ($M_v \sim 5.1$) possible, but our apparent magnitude
limit for the observations is $m_v =24.0$. The central stellar density
of the halo is expted to be $\sim 1\%$ that of the disk component, but
the number density of halo stars falls off as $R^{-3.5}$ whereas the
stellar density in the disk decreases as $n \sim exp(-R/R_0)$, where
$R_0\sim 3kpc$ (also known as the disk scale length). \\

a.) What is the maximum distance (in kpc) out to which we can observe
the stars in question?\\

\textbf{Solution:}

The maximum distance assuming no dust would be:

\begin{equation}
d = 10^{18.9/5+1} = 60.25 kpc
\end{equation}

b.) At that distance, what is the probability that the star is a
member of the stellar halo rather that the stellar disk? What is the
probability in the solar neighborhood of observing a halo star from
among all G5 spectral type stars?

\textbf{Solution}:

The probability of observing a halo star would be:

\begin{equation}
P = 1 - 1/(60.25^{-3.5}/exp(-60.25/3)) = 99.6 \%
\end{equation}

While in the solar neighborhood the probability would be $0.4 \%$.\\

c.) Now put some dust in the disk so that the interstellar reddening
aling the line of sight in thedisk is $E(B-V)=0.5(d/kpc)$, where $d$
is the distance along the line of sight to the star. Assuming a Milk
Way like extinction curve in the disk, how does the answer to part
(a) change if we have the same limiting $m_v=24$ for the
observation?\\

\textbf{Solution:}

The distance modulus equation have to be corrected by the extinction,
this is:

\begin{equation}
d = 10^{(m_v-M_v-A_v)/5 + 1}
\end{equation}

Where $A_v=1.6d$ for a Milky Way like extinction curve, finally the
distance would be $d=3.75kpc$.\\

5.) Salpeter's Initial Mass Function (IMF) is of the form:

\begin{equation}
\Phi(M) \sim M^{-(1+x)}
\end{equation}

By considering only stars more massive than 1 solar mass (whose life
times are shorter than the age of the Galaxy) and stellar luminosities
$L\sim M^4$, find the slope $x$ such that equal numbers of stars are
seen in a homogeneous isotropic region within equal logarithmic ranges
of luminosity. What type of star dominates the counts if the slope $x$
is flatter than this value? \\

\textbf{Solution:}

First we can compute the Luminosity function from the IMF as follows:

\begin{equation}
\Phi(L) = \Phi(M)\dfrac{dM}{dL} = \dfrac{L^{-(1+x/4)}}{4}
\end{equation}

Where we use the realtion between Mass and Luminosity to get $dM/dL =
\dfrac{L^{-3/4}}{4}$.

The number of stars seen in an equal logarithmic range of luminosity
is:\\

\begin{equation}
N = \int_{L_0}^{L_1} N= \int_{L_0}^{L_1} \dfrac{L^{-(1+x/4)}}{4} dL
\end{equation}

For different ranges of luminosity $N$ is going to be constant only in
the case of $x=0$, if $x \neq 0$ $N$ is going to depend on the
luminosity range. In the case of $x=0$ the number of stars goes as
$N\sim log(M)$ the massive stars dominates, which is not what it has
been observed.\\

6.) The star formation history of a stellar population is often
represented by an exponential decay from the initial burst, viz:
$\Psi(t)\sim exp(-t/\tau)$ where $\tau$ is some time constant. If the
IMF $\Phi(M)$ is invariant, obtai an expression for the observed number
of stars of a given mass at time $t$ in terms of tis main sequence
lifetime. Comment briefly in the differences you would expect to see
in the H-R diagrams of a population where $\tau=0.1 Gyr$ and
$\tau=\infty$ for a population viewed after $12Gyr$.\\

\textbf{Solution:}

The number of stars given that the IMF is invariant in going to be:

\begin{equation}
N \sim \int_{t=0}^{t'} \Psi dt = \int_{t=0}^{t'} exp(-t/\tau) = -\tau
exp(-t'/\tau) + \tau
\end{equation}

At $\tau=0.1$ most of the stars most be old and they should be located close to
the turn over, the red giant branch etc. While in the case of the $\tau=\infty$ most of the
stars most be young and they should be in the main sequence.

\end{document}
