\documentclass[14pt]{article}
\usepackage{amsmath}
\usepackage{listings} % For writing code see http://ctan.org/pkg/listings
\usepackage{graphicx}
\usepackage{float}

\title{HW \#1 ASTR 540}
\author{Nico Garavito}
\begin{document}

\maketitle

1. The declination of a star is $42^{\circ}57'$ N and its proper
motion components are $\mu_{\alpha}=-0.0374''$, $\mu_{\delta}=1.21"$.
Calculate its total proper motion. If the spectrum reveals a blueshift
of $7.6 kms^{-1}$ and the parallax is $0.376"$, calculate its space
velocity relative tp the Sun and its total proper motion at the time
of closest approach. \\

\textbf{Solution:}

\begin{table}[h]
\centering
\begin{tabular}{c}
\hline
$\delta = 42^{\circ} 57' = 42.95^{\circ}$ \\
$\mu_{\alpha} = -0.0374''$ \\
$\mu_{\delta} = 1.21''$\\
\hline
\end{tabular}
\caption{\label{tab:1}}
\end{table}

From the proper motion components resumed in table \ref{tab:1}
we can compute the total proper motion as follows:

\begin{equation}
\mu_{\delta} = \mu cos(\theta)
\end{equation}

\begin{equation}
\mu = \dfrac{\mu_{\delta}}{cos(\theta)}
\end{equation}

\begin{equation}
\mu_{\alpha} = \dfrac{\mu sin(\theta)}{cos(\delta)}
\end{equation}

\begin{equation}
tan(\theta)=\dfrac{cos(\delta) \mu_{\alpha}}{\mu_{\delta}}
\end{equation}

\begin{equation}
\theta = 1.3035^{\circ}
\end{equation}

And the total proper motion would be:

\begin{equation}
\mu = 1.21 ''/yr
\end{equation}

With the parallax we can compute the distance as follows:

\begin{equation}
d = \dfrac{1AU}{tan(0.376'')} = 180 AU /(0.376'' * \pi) = 548576.612 AU
\end{equation}

The space velocity relative to the Sun would be:

\begin{equation}
V_t = \mu \cdot d = \dfrac{1.21}{3600} \dfrac{\pi}{180} 548576.12 AU
/yr = 15.255 km/s
\end{equation}

\begin{equation}
V = (V_t^2 + V_r^2)^{(1/2)} = (-0.46^2 + 7.6^2)^{1/2} = 17.03 km/s
\end{equation}

Finally the total proper motion at the time of closest approach would
be:

\begin{equation}
\mu = \dfrac{V_t}{d}
\end{equation}

Where the distance of closest approach $d_0$ would be:

\begin{equation}
d_0 = d \cdot sin(\theta) = d \cdot sin(arctan(V_t/V_r)) =  491015.38
 AU
\end{equation}

And the total proper motion at this time $t_0$ is going to be:

\begin{equation}
\mu = 17.03 km/s / 491015.38 AU = 1.5''/yr
\end{equation}

2. Smoot et al (1992) found a dipole anisotropy in COBE measurements
of the microwave background such that the background is higher by
$3.36mK$ (c.f. 2.73K average) in the direction $\alpha=11h09m$
,$\delta=-7^{\circ}$. By subtracting Galatic rotational motion of
the solar neighborhood ($v=220km/s$), determine the direction (in
celestial coordinates) and peculiar speed of the Milky Way with
respect to the microwave background. The IAU approved celestial
coodiantes of the North Galactic Pole is $192^{\circ}.85948,
27^{\circ}.12825$ and the Galactic longitude of th north celestial pole
is $123.932$.\\

\textbf{Solution:}

Using the difference in temperature of the microwave background and
the dipole anisotropy the relative velocity of the Solar neighbor with 
respect to the CMB can be calculated as:

\begin{equation}
\dfrac{T'}{T} = \left( \dfrac{c-v}{c+v}\right)^{1/2}
\end{equation}

Where $T=2.73K$ and $T'=T+3.36mK$. After some algebra I found:

\begin{equation}
v = \dfrac{c\left (1 - \left (\dfrac{T'}{T}\right )^2 \right )}
{1 + \left (\dfrac{T'}{T}\right )^2} = -369.23 km/s
\end{equation}

The direction of the CMB anisotropy in Galactic coordinates can be
computed using the coordinates transformation relation:

\begin{equation}
l=263.22
b = 47.81
\end{equation}

Finally the component of the solar neighbor rotational velocity
$v_{\odot}$ in the
direction of the CMB dipole anisotropy $v_{\odot, CMB}$ is:

\begin{equation}
v_{\odot, CMB} = v_{\odot} cos(\theta)cos(\phi) = -146.71 km/s
\end{equation}

Finally the peculiar velocity of the Milky Way with respect to the
CMB is $v_{MW, CMB} = -146.71+(-369.23) = -515.94 km/s$

3. Bahcall \& Soneira's (1980) assumed that the luminosity density of
the Galactic disk as a function of radius canbe fit by an exponential
function. Using near-infrared (K-band) spacelab 2 data, they
determined a central disk luminosity of $1208  L_{odot} pc^{-2}$ and a
disk scale length $R_d=2.7 kpc$. If the distance from the Sun to the
Galactic center is $R_0=8kpc$, estimate the projected surface
brightness $\mu_K$ (in mags arcsec$^{-2}$) of the Galactic disk in
the solar neighborhood as viewd by an external observer. The absolute
magnitude of the Sun is $M_K=+3.28$.

\textbf{Solution:}

\begin{equation}
M_{bol, gal} = M_B - 2.5Log_{10}(L_{gal}/L_{\odot}) = 5.48 -
2.5Log_{10}(1.4\times 10^{10}) = -19.58
\end{equation}

4.

\end{document}
